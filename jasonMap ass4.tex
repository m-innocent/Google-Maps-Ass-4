\documentclass[7pt]{article}
\raggedright
\parindent=0in \parskip=8pt
\usepackage{graphicx}
\usepackage[margin=1in]{geometry} % 1 inch margins all around
\begin{document}

\begin{Huge}
\begin{center}
\begin{normalsize}
\textbf{MAKERERE \includegraphics[scale=0.5]{logo} UNIVERSITY }\\

\textbf{COLLEGE OF COMPUTING AND INFORMATION SCIENCES} \\

\textbf{BACHELOR OF SCIENCE IN COMPUTER SCIENCE} \\
\textbf{YEAR 2} \\
\textbf{BIT 2207 RESEARCH METHODOLOGY} \\
\textbf{Course Work: Assignment 4}\\
\end{normalsize}
\end{center}
\end{Huge}

\begin{center}
\begin{tabular}{l l l}
\textbf{NAME}  & \textbf{REGISTRATION NUMBER} & \textbf{STUDENT NUMBER} \\
MURUHURA INNOCENT & 16/U/7611/PS & 216013512 \\
\end{tabular}

\paragraph{•}
\textbf{Lecturer}: Mr. Earnest Mwebaze
\end{center}

\newpage

\title{THE REASONS WHY GOOGLE MAPS IS THE MOST POPULAR GPS LOCATION APPLICATION}
\author{MURUHURA INNOCENT}      
\renewcommand{\today}{}

\maketitle

\section*{INTRODUCTION}

\paragraph{•}
Google Maps is a web mapping service developed by Google. It offers satellite imagery, street maps, 360° panoramic views of streets (Street View), real-time traffic conditions (Google Traffic), and route planning for traveling by foot, car, bicycle (in beta), or public transportation.
\section*{LITERATURE REVIEW}
\paragraph{•}
Google Maps began as a C++ desktop program designed by Lars and Jens Eilstrup Rasmussen at Where 2 Technologies. In October 2004, the company was acquired by Google, which converted it into a web application. After additional acquisitions of a geospatial data visualization company and a realtime traffic analyzer, Google Maps was launched in February 2005. The service's front end utilizes JavaScript, XML, and Ajax. 
In 2012, Google reported having over 7100 employees and contractors directly working in mapping.
\cite{sharma}
\subsection*{Directions}
\paragraph{•}
Google Maps provides a route planner,allowing users to find available directions through driving, public transportation, walking, or biking.
\paragraph{•}
\subsection*{Implementation}
\paragraph{•}
Like many other Google web applications, Google Maps uses JavaScript extensively. As the user drags the map, the grid squares are downloaded from the server and inserted into the page. When a user searches for a business, the results are downloaded in the background for insertion into the side panel and map; the page is not reloaded. Locations are drawn dynamically by positioning a red pin (composed of several partially transparent PNGs) on top of the map images. A hidden IFrame with form submission is used because it preserves browser history. The site also uses JSON for data transfer rather than XML, for performance reasons. These techniques both fall under the broad Ajax umbrella. The result is termed a slippy map and is implemented elsewhere in projects such as OpenLayers.
\cite{Likhith P}
\paragraph{•}
\subsection*{Extensibility and customization}
\paragraph{•}
As Google Maps is coded almost entirely in JavaScript and XML, some end users have reverse-engineered the tool and produced client-side scripts and server-side hooks which allowed a user or website to introduce expanded or customized features into the Google Maps interface. Using the core engine and the map/satellite images hosted by Google, such tools can introduce custom location icons, location coordinates and metadata, and even custom map image sources into the Google Maps interface. The script-insertion tool Greasemonkey provides a large number of client-side scripts to customize Google Maps data.
\section*{Meet the woman behind Google Maps}
\subsection*{By Caroline Fairchild}
\paragraph{•}
Today marks the 10th anniversary of Google Maps, the revolutionary navigation product that changed the way we all think about getting from A to B.
Back in 2005, Google Maps (goog, +1.46 percent) was simply a way for users to get basic directions. Since, it’s aggressively evolved and the product now allows Googlers to pinpoint what coffee shops are down the road from their offices as easily as they can explore the pyramids of Egypt. Amanda Leicht Moore joined the Google Maps team as an intern in 2007. Today, she’s the division’s lead product manager. To celebrate the 10th birthday of a product that helps so many lost users find their bearings, Fortune talked to Moore about what’s in store for Google Maps’ future.\cite{Caroline}
\subsubsection*{Questions by Fortune:}
\paragraph{•}  
Q). Remind us what the world of navigation looked like before Google Maps came out?
\subsubsection*{Answers by Amanda Leicht Moore: }
\paragraph{•}  
 A). It’s kind of crazy, 10 years ago before Google Maps came out, people were really reliant on paper maps and guide books. Anytime you went to a new place, you had to stop at the gas station and buy [a map]. Now, everyone has a map of the world in their pocket and you are never truly lost. It fundamentally changes how people think about exploring their world and moving about.\cite{Caroline}
\subsubsection*{Q). What’s next for Google Maps?}
\paragraph{•} 
 A). We’re been thinking a lot about how we can make people’s day-to-day lives easier. There are a lot of things in your day-to-day life that are stressful. Is my train going to be on time? Which highway should I take to work? Where should I go to lunch? We think we can make these decisions easier for people and make you feel more comfortable about where you are going.\cite{Caroline}
\subsubsection*{Q). There are not lots of navigations apps out there. Why is Google uniquely positioned to give people this information?}
\paragraph{•}
A). Google does imagery really well. You can travel the world from the palm of your hand. That is a really exciting part of Google Maps. In the last year, we've also focused a lot on helping you explore the world around you. So, that means what places are nearby and where you should grab a bite to eat for dinner.
Given gender stereotypes around women being bad at directions, do you find that people are surprised that you are a woman?
To be honest, I really do have a terrible sense of direction, but I feel that is an asset. I need to use my product all the time. I rely on Google Maps to not be lost, so that is actually a huge benefit. (Laughs)\cite{Caroline} 
\subsubsection*{Q). 10 years from today, where do you think Google Maps will be?}
\paragraph{•} 
A). It’s really interesting to see how maps have evolved–from just showing you a map and you having to parse through the map and get the information out of it to [a product] that really helps you understand the world. We are going to see more of that, where we can tell you how to go to a place and you don’t even have to look it up. In 10 years, I don’t even know if we’ll have smartphones so it’s kind of hard to imagine. I hope 10 years from now we can definitively say that we completely solved the problem of ever getting lost\cite{Caroline}

\begin{thebibliography}{9}
\bibitem{sharma} Shivani Sharma \textit{Quora}, 
Internet: www.quora.com/Technology Writer (2016-present), Answered Apr 28, 2017.
\bibitem{likhith} Likhith P. \textit{Fullcircle design} 
Internet: https://www.quora.com/Local guide for google maps, Answered Apr 22, 2017.
\bibitem{Caroline} Caroline Fairchild. \textit{Meet the woman behind Google Maps},
 Internet: www.http://fortune.com/2015/02/09/google-maps-birthday//By Caroline Fairchild 
February 9, 2015 

\end{thebibliography}

\end{document}